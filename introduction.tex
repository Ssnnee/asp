\chapter*{Introduction}
\addcontentsline{toc}{chapter}{Introduction}
Aujourd’hui, les technologies informatiques influencent considérablement notre
vie quotidienne. La convergence des plateformes web et mobiles a remodelé la
façon dont nous nous connectons, partageons et préservons nos informations.
En tant que discipline, la généalogie est ancrée dans la préservation et la
transmission de l’héritage familial. Avec l’avènement des technologies modernes,
il devient impératif d’explorer de nouvelles perspectives pour faciliter la
création et le partage d’arbres généalogiques. Dans le cadre de notre projet pour
l’obtention de la licence professionnelle en filière informatique de gestion,
option génie logiciel, au sein de l'\ac{E.S.G.A.E} , nous nous sommes engagés dans l’étude et la mise
en œuvre d’une plateforme visant à simplifier et à enrichir l’expérience de
reconstitution et de partage d’arbres généalogiques.

Nous avons choisi comme thème de projet \say{\projettheme}. Cette initiative vise à répondre à un besoin croissant : la préservation
et le partage de l’héritage familial à l’ère numérique.
Notre projet s’articule autour de trois objectifs principaux :
1. Faciliter la création collaborative d’arbres généalogiques : nous souhaitons
fournir une plateforme conviviale où les membres de la famille peuvent collaborer
efficacement pour construire et partager leurs arbres généalogiques.
2. Simplifier le partage des informations généalogiques : nous souhaitons créer
une interface intuitive sur les plateformes web et mobiles, permettant aux utilisateurs
de partager facilement leurs arbres généalogiques avec leurs proches.
3. Promouvoir la découverte de son histoire ou de son héritage familial.
Notre plateforme est destinée à tous ceux qui souhaitent explorer et partager
leur lignée familiale, que ce soit des novices ou des professionnels de la généalogie.
Pour mener à bien ce projet, nous avons opté pour une approche méthodologique agile :
la méthode agile \ac{2TUP}.
Cette méthode, axée sur la collaboration et l’itération, nous permettra de développer
la plateforme de manière incrémentale, en nous adaptant aux besoins changeants des utilisateurs.

Tout au long du processus, nous mettrons en œuvre les meilleures pratiques de
développement logiciel pour garantir la qualité, la sécurité et la performance
de notre solution. Notre objectif est de fournir une expérience enrichissante à
toute personne désireuse de découvrir l’histoire de sa famille et de la partager avec ses proches.

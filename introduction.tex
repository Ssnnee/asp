\chapter*{Introduction}
\addcontentsline{toc}{chapter}{Introduction}
\label{chap:introduction}
Au cœur de l’ère numérique actuelle, l’humanité fait face à un défi majeur :
exploiter la puissance des technologies modernes pour transformer la création
et le partage des arbres généalogiques. À l’intersection de la généalogie et
du développement de logiciels, cette question nous invite à repenser la
manière dont nous conservons et transmettons notre héritage familial.
Intégrer les innovations numériques est essentiel pour explorer de nouvelles
perspectives, facilitant ainsi la reconstitution et la diffusion des arbres
généalogiques, rendant ce processus plus accessible et interactif pour tous.

Dans le cadre de notre projet pour l’obtention de la licence professionnelle en
filière informatique de gestion, option génie logiciel, à l’\ac{ESGAE}, nous nous
sommes engagés à analyser et développer une plateforme visant à apporter une
réponse à ce défi. Nous avons choisi comme thème
\textbf {\og Étude et mise en œuvre d’une plateforme web et mobile pour la
création collaborative et le partage d’arbres généalogiques à Mazala-Firm \fg}.
Cette initiative répond à un besoin croissant : la préservation et le partage de l’héritage
familial à l’ère numérique.

Notre projet s’articule autour de trois objectifs principaux :
\begin{itemize}
  \item Simplifier la construction participative d’arbres généalogiques :
    nous souhaitons fournir une plateforme conviviale où les membres de
    la famille peuvent coopérer efficacement pour bâtir et partager leurs
    arbres généalogiques.

  \item Faciliter le partage des informations généalogiques : nous désirons
    créer une interface intuitive sur les plateformes web et mobiles,
    permettant aux utilisateurs de partager facilement leurs arbres
    généalogiques avec leurs proches.

  \item Promouvoir la découverte de l’histoire et de l’héritage familial :
    notre plateforme est destinée à tous ceux qui aspirent à explorer et
    partager leur lignée familiale, que ce soit des novices ou des
    professionnels de la généalogie.
\end{itemize}

Pour mener à bien ce projet, nous avons opté pour une démarche méthodologique
rigoureuse en utilisant la méthode objet.

Tout au long du processus, nous mettrons en œuvre les meilleures pratiques de
développement logiciel pour garantir la qualité, la sécurité et la performance
de notre solution. Notre objectif est de fournir une expérience enrichissante à toute
personne désireuse de découvrir l’histoire de sa famille et de la
partager avec ses proches.

Ce document dépeint le travail accompli et se compose de trois parties distinctes.
La première présentera l’environnement de travail en décrivant la structure qui
nous a accueillis, ainsi que l’idée qui nous a été confiée, son contexte et ses
enjeux. Dans la deuxième, nous approfondirons l’analyse, la conception et la
mise en œuvre du projet, puis exposerons les outils et technologies utilisés.
Enfin, la troisième traitera de la gestion du projet, de son évaluation et des perspectives.

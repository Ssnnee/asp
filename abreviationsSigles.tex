\chapter*{Abréviation et Sigles}
\addcontentsline{toc}{chapter}{Abréviation et Sigles}
% \thispagestyle{nohede
\begin{acronym}

  \acro{ESGAE}{École Supérieure de Gestion et d’Administration des Entreprises}
  \acro{LPGL}{Licence Professionnelle en Génie Logiciel}
  \acro{2TUP}{Two Tracks Unified Process}
  \acro{UP}{Unified Process ou Processus Unifie}
  \acro{UI}{User Interface ou interface utilisateur en français}
  \acro{UX}{User Experience ou experiance utilisateur en Français}
  \acro{UML}{Unified Modeling Language}
  \acro{API}{Application Programming Interface}
  \acro{RAM}{Random Access Memory ou Mémoire vive en français}
  \acro{Go}{Giga octet}
  \acro{SSD}{Solid State Drive ou Disque à état solide en français}
  \acro{HD}{High Definition ou Haute Définition en français}
  \acro{sRGB}{Standard Red Green Blue ou Rouge Vert Bleu en français}
  \acro{RAID}{Redundant Array of Independent Disks ou Ensemble redondant de disques indépendants en français}
  \acro{NVMe}{Non-Volatile Memory Express ou Express de mémoire non volatile en français}
  \acro{VPN}{Virtual Private Network ou Réseau Privé Virtuel en français}
  \acro{XP}{Extreme Programming}
  \acro{ORM}{Object Relational Mapping}
  \acro{SGBD}{Système de Gestion de Base de Données}
  \acro{SQL}{Structured Query Language}
  \acro{CRUD}{Create, Read, Update, Delete}
  \acro{API}{Application Programming Interface}
  \acro{HTTP}{Hypertext Transfer Protocol}
  \acro{HTTPS}{Hypertext Transfer Protocol Secure}
  \acro{URL}{Uniform Resource Locator}
  \acro{SSR}{Server Side Rendering}
  \acro{SSG}{Static Site Generation}
  \acro{SEO}{Search Engine Optimization}
  \acro{OTA}{Over The Air}
  \acro{CI/CD}{Continuous Integration/Continuous Deployment}
  \acro{JWT}{JSON Web Token}
  \acro{RPC}{Remote Procedure Call}
  \acro{tRPC}{Typed RPC}
  \acro{TLS}{Transport Layer Security}



\end{acronym}


\begin{table}[h!]
    \centering
    \begin{tabular}{|c|m{11cm}|}
        \hline
        \textbf{Abréviation ou sigle} & \textbf{Signification} \\
        \hline
        ESGAE & École Supérieure de Gestion et d’Administration des Entreprises \\
        \hline
        LPGL & Licence Professionnelle en Génie Logiciel \\
        \hline
        2TUP & Two Tracks Unified Process \\
        \hline
        UP & Unified Process ou Processus Unifié \\
        \hline
        UI & User Interface ou interface utilisateur en français \\
        \hline
        UX & User Experience ou expérience utilisateur en français \\
        \hline
        UML & Unified Modeling Language \\
        \hline
        API & Application Programming Interface \\
        \hline
        RAM & Random Access Memory ou Mémoire vive en français \\
        \hline
        Go & Giga octet \\
        \hline
        SSD & Solid State Drive ou Disque à état solide en français \\
        \hline
        HD & High Definition ou Haute Définition en français \\
        \hline
        sRGB & Standard Red Green Blue ou Rouge Vert Bleu en français \\
        \hline
        RAID & Redundant Array of Independent Disks ou Ensemble redondant de disques indépendants en français \\
        \hline
        NVMe & Non-Volatile Memory Express ou Express de mémoire non volatile en français \\
        \hline
        VPN & Virtual Private Network ou Réseau Privé Virtuel en français \\
        \hline
        XP & Extreme Programming \\
        \hline
        ORM & Object Relational Mapping \\
        \hline
        SGBD & Système de Gestion de Base de Données \\
        \hline
        SQL & Structured Query Language \\
        \hline
        CRUD & Create, Read, Update, Delete \\
        \hline
        HTTP & Hypertext Transfer Protocol \\
        \hline
        HTTPS & Hypertext Transfer Protocol Secure \\
        \hline
        URL & Uniform Resource Locator \\
        \hline
        SSR & Server Side Rendering \\
        \hline
        SSG & Static Site Generation \\
        \hline
        SEO & Search Engine Optimization \\
        \hline
        OTA & Over The Air \\
        \hline
        CI/CD & Continuous Integration/Continuous Deployment \\
        \hline
        tRPC & Typed RPC \\
        \hline
        RPC & Remote Procedure Call \\
        \hline
        TLS & Transport Layer Security \\
        \hline
    \end{tabular}
    \caption{Abréviation ou sigle et significations}
\end{table}

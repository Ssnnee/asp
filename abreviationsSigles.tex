\chapter*{Abréviation et Sigles}
\addcontentsline{toc}{chapter}{Abréviation et Sigles}
% \thispagestyle{nohede
\begin{acronym}

  \acro{ESGAE}{École Supérieure de Gestion et d’Administration des Entreprises}
  \acro{LPGL}{Licence Professionnelle en Génie Logiciel}
  \acro{2TUP}{Two Tracks Unified Process}
  \acro{UP}{Unified Process ou Processus Unifie}
  \acro{UI}{User Interface ou interface utilisateur en français}
  \acro{UX}{User Experience ou experiance utilisateur en Français}
  \acro{UML}{Unified Modeling Language}
  \acro{API}{Application Programming Interface}
  \acro{RAM}{Random Access Memory ou Mémoire vive en français}
  \acro{Go}{Giga octet}
  \acro{SSD}{Solid State Drive ou Disque à état solide en français}
  \acro{HD}{High Definition ou Haute Définition en français}
  \acro{WQHD}{Wide Quad High Definition ou Ultra HD en français}
  \acro{FHD}{Full High Definition ou Full HD en français}
  \acro{UHD}{Ultra High Definition ou Ultra HD en français}
  \acro{sRGB}{Standard Red Green Blue ou Rouge Vert Bleu en français}
  \acro{RAID}{Redundant Array of Independent Disks ou Ensemble redondant de disques indépendants en français}
  \acro{NVMe}{Non-Volatile Memory Express ou Express de mémoire non volatile en français}
  \acro{VPN}{Virtual Private Network ou Réseau Privé Virtuel en français}
  \acro{XP}{Extreme Programming}
  \acro{ORM}{Object Relational Mapping}
  \acro{SGBD}{Système de Gestion de Base de Données}
  \acro{SQL}{Structured Query Language}
  \acro{CRUD}{Create, Read, Update, Delete}
  \acro{API}{Application Programming Interface}
  \acro{HTTP}{Hypertext Transfer Protocol}
  \acro{HTTPS}{Hypertext Transfer Protocol Secure}
  \acro{URL}{Uniform Resource Locator}
  \acro{SSR}{Server Side Rendering}
  \acro{SSG}{Static Site Generation}
  \acro{SEO}{Search Engine Optimization}
  \acro{OTA}{Over The Air}
  \acro{CI/CD}{Continuous Integration/Continuous Deployment}
  \acro{JWT}{JSON Web Token}
  \acro{tRPC}{Typed RPC}
  \acro{RPC}{Remote Procedure Call}
  \acro{TLS}{Transport Layer Security}



\end{acronym}


% \begin{tabularx}{\textwidth}{|l|l|}
%     \hline
%     \textbf{Cas d'utilisation} & \textbf{Priorité}
%     ES
%   \end{tabularx}

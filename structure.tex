\part{Présentation Générale de \firm}
\label{part:presentation}
\chapter{Structure d’Accueil et le Sujet}
\section{Structure d’Accueil}
\subsection{Historique}

\mazf est une agence de développement spécialisée dans la création de solutions
numériques novatrices. Créée en 2021 par Inefable KOUMBA, un passionné des nouvelles
technologies, \mazf s'est rapidement établie comme un acteur de premier plan dans ce domaine.
L'histoire de \mazf débute avec Inefable KOUMBA, un ingénieur informatique
visionnaire qui avait pour but de fonder une agence proposant des services complets
en développement web. Fort de son expertise et de sa passion pour l’innovation,
il rassemble une équipe de talents aux compétences variées, allant de l’analyse
de données au design UX/UI en passant par le développement logiciel.
Dès ses débuts, \mazf se distingue par son approche holistique du développement
web. L’agence comprend rapidement que la clé du succès réside dans la capacité à
transformer les données en informations exploitables. Ainsi, l’analyse des données
devient l’un des piliers fondamentaux de l’activité de l’agence. Grâce à une
expertise avancée en analyse de données, \mazf aide ses clients à tirer
parti de leurs données pour prendre des décisions éclairées et stratégiques.
Parallèlement, l’agence met un point d’honneur à offrir des expériences utilisateur
exceptionnelles à travers ses services de conception UI/UX. En combinant esthétique
et fonctionnalité, les designers de \mazf façonnent des interfaces intuitives
et attractives, garantissant ainsi des interactions mémorables avec les produits
numériques de ses clients. Le développement logiciel sur mesure est une autre
spécialité de \mazf. Que ce soit pour la création d’applications mobiles,
de plateformes web complexes ou de logiciels d’entreprise, l’équipe de développeurs
de l’agence excelle dans l’art de concevoir des solutions sur mesure répondant
parfaitement aux besoins spécifiques de chaque client.

\subsection{Missions}
Les missions de \firm sont mutliples et s’articulent autour de plusieurs axes. L’entreprise
est engagée dans la transformation de données en informations exploitables, dans
la création d’expériences utilisateur simples et intuitives, dans le développement de solutions
logicielles personnalisées, dans la fourniture de conseils stratégiques et d’audits
informatiques pertinents, dans la gestion fiable des systèmes et réseaux informatiques,
ainsi que dans la proposition de solutions de stockage sur le nuage sécurisées et flexibles.

Voici les missions de \firm :
\begin{enumerate}
  \item \textbf{Analyse des données :} Notre mission est de transformer les données brutes
    en informations exploitables. Nous utilisons une expertise avancée en analyse
    de données pour aider nos clients à comprendre, interpréter et exploiter efficacement
    les données pour prendre des décisions stratégiques et éclairées.
  \item \textbf{UI \& UX Design :} nous nous engageons à créer des expériences utilisateur mémorables
    en combinant l’esthétique et la fonctionnalité. Notre équipe de concepteurs
    d’interface utilisateur/expérience utilisateur travaille en étroite collaboration
    avec nos clients pour concevoir des interfaces intuitives, attrayantes et ergonomiques,
    garantissant ainsi une satisfaction utilisateur optimale.
  \item \textbf{Développement logiciel :} notre objectif est de créer des solutions sur
    mesure qui répondent parfaitement aux besoins spécifiques de chaque client.
    Que ce soit pour le développement d’applications mobiles, de plateformes Web
    ou de logiciels d’entreprise, notre équipe de développeurs s’engage à fournir
    des solutions innovantes et de haute qualité.
  \item \textbf{Conseil et audit informatique :} nous proposons des conseils stratégiques
    et des audits informatiques afin d’aider nos clients à optimiser leurs systèmes
    et leurs investissements technologiques. Notre mission est d’accompagner nos
    clients dans leur transformation digitale en leur fournissant des recommandations
    avisées et des solutions adaptées à leurs besoins spécifiques.
  \item \textbf{Administration des systèmes et des réseaux informatiques :} nous nous
    engageons à assurer une gestion efficace et optimale des systèmes et des
    réseaux informatiques de nos clients. Notre équipe d’administrateurs système
    expérimentés travaille en coulisses pour garantir la fiabilité, la sécurité et
    la performance des infrastructures informatiques de nos clients.
  \item \textbf{Stockage des données sur le nuage :} Notre mission est de fournir des solutions
    de stockage sur le nuage sécurisées et flexibles. En collaboration avec les
    principaux fournisseurs de services infonuagiques, nous offrons à nos clients
    un accès simple et sécurisé à leurs données, où qu’ils se trouvent.
\end{enumerate}

\subsection{Organigramme Général}
L’organigramme de \firm est bien structuré et se présente comme suit :

\begin{figure}[h]
  \centering
  \begin{forest}
    for tree={
      draw,
      align=center,
      minimum height=2em,
      minimum width=4em,
      grow'=0,
      parent anchor=south,
      child anchor=north,
      edge path={
        \noexpand\path[\forestoption{edge}]
        (\forestOve{\forestove{@parent}}{name}.parent anchor) -- +(0,-5pt) -|
        (\forestove{name}.child anchor)\forestoption{edge label};
      },
    },
    forked edges,
    [ Directeur général
        [ Sécrétaire ]
        [Responsable comptable]
        [Responsable marketing
          [Community manager]
        ]
        [Responsable technique
          [Equipe de développement]
          [Equipe de designers]
        ]
    ]
    \node [draw, fit=(current bounding box.south east) (current bounding box.north west), inner sep=10pt] {};
  \end{forest}
  \caption{Organigramme de \firm }
\end{figure}

\subsection{Attributions des structures}
L’organigramme ci-dessus illustre le rôle crucial que jouent chaque poste et
département au sein de \firm. Voici une brève description de chaque
structure et de ses principales responsabilités :
\begin{enumerate}
  \item \textbf{Directeur général :} Chef d’entreprise, responsable de la gestion globale,
    des décisions stratégiques et de la supervision des départements et des employés.

  \item \textbf{Secrétaire :} assiste les dirigeants et les cadres supérieurs dans leurs
    tâches administratives, notamment la gestion du courrier et des réunions.

  \item \textbf{Responsable comptable :} gère les finances de l’entreprise, y compris la
    tenue des livres comptables et la préparation des rapports financiers.

  \item \textbf{Responsable marketing :} planifie, mets en œuvre et gère les stratégies
    marketing, y compris la publicité et les relations publiques.

  \item \textbf{Responsable technique :} supervise les aspects techniques de l’entreprise,
    notamment l’infrastructure informatique et les processus de développement.

  \item \textbf{Community manager :} gère et anime les communautés en ligne, interagis
    avec les clients et crée du contenu pour engager la communauté.

  \item \textbf{Équipe de développement :} conçois, développe et maintiens les logiciels
    et les applications de l’entreprise et des produits.

  \item \textbf{Équipe designer :} responsable de la conception visuelle des produits,
    incluant la conception graphique et l’interface utilisateur.

\end{enumerate}

\subsection{Situation informatique}
\firm dispose d’un département informatique dédié, supervisé par le Responsable
technique. Ce département est structuré en deux équipes principales :
\begin{itemize}

  \item \textbf{Équipe de développement:} conçois et développe des solutions logicielles
    personnalisées en collaboration avec les clients.
  \item \textbf{Équipe de design:} crée des interfaces utilisateur attrayantes et intuitives
    pour les produits numériques de l’entreprise.
\end{itemize}

\subsubsection{Personnel informatique}
L’équipe du service informatique comprend une équipe talentueuse composée de :
\begin{itemize}
  \item un chef de Service technique
  \item des développeurs logiciels
  \item des designers UX/UI
  \item des ingénieurs et techniciens informatiques spécialisés dans
  \item le développement et la maintenance des solutions logicielles et des interfaces utilisateur.
\end{itemize}


Cette structure organisationnelle garantit que \firm dispose des ressources
et des compétences nécessaires pour répondre aux besoins technologiques de ses
clients et fournir des solutions informatiques de haute qualité et innovantes.


\subsubsection { Matériels informatiques }
L’efficacité de \firm dépend étroitement de notre matériel informatique.
En tant qu’entreprise spécialisée dans le développement web et le design, nous
comprenons l’importance d’un équipement de qualité pour soutenir nos équipes dans
la réalisation de projets innovants. Ci-dessous, nous présentons les principaux
composants de notre infrastructure informatique, démontrant ainsi notre engagement
envers des solutions performantes et une satisfaction client optimale.

Ainsi, \firm dispose de l’équipement informatique suivant :

\begin{itemize}
  \item Des ordinateurs portables performants dotés de processeurs Intel Core 
    i7 ou i9, ou AMD Ryzen 7 ou 9, avec au moins 12 Go de RAM et des disques
    SSD de 512 Go à 1 To. Ils conviennent parfaitement aux développeurs et aux
    concepteurs, que ce soit au bureau ou sur le terrain.

  \item Stations de travail  puissantess pour les tâches de développement et
    de conception les plus exigeantes, dotés de processeurs Intel Xeon ou AMD
    Threadripper avec 32 à 64 Go de RAM et des cartes graphiques dédiées comme
    les NVIDIA Quadro ou AMD Radeon Pro pour accélérer les rendus et la visualisation.

  \item Écrans de haute résolution de 27 à 32 pouces, avec une résolution
    minimale de 2560x1440 (Ultra HD) ou 3840x2160 (4K Ultra HD) et une couverture de
    l'espace colorimétrique sRGB ou Adobe RGB à 99%, pour permettre aux concepteurs de visualiser leurs créations dans les moindres détails.

  \item Tablettes graphiques professionnelles comme les Wacom Cintiq ou
    Huion Kamvas Pro, avec des tailles d'écran de 16 à 24 pouces et une
    résolution de 1920x1080 (Full HD) ou supérieure, permettant de dessiner et
    de retoucher des images avec précision.

  \item Serveurs de hautes performances équipés de processeurs Intel Xeon ou
    AMD EPYC, avec au moins 64 Go de RAM et des options de stockage RAID avec
    des disques SSD ou NVMe, garantissant une disponibilité et une sécurité
    maximales pour les services en ligne de Mazala-Firm.

  \item Équipements réseau  tels que des routeurs compatibles Wi-Fi 6, des
    commutateurs Ethernet gigabit ou 10 gigabit, et des points d'accès Wi-Fi
    couvrant l'ensemble du bâtiment pour assurer une connexion réseau fiable et
    sécurisée.

  \item Accessoires comme  Claviers ergonomiques et souris de précision de
    marque HP, casques audio, et autres accessoires informatiques pour
    permettre à chaque membre de l’équipe de travailler confortablement et
    efficacement.

\end{itemize}



\subsubsection   { Logiciels informatiques }
\firm utilise une gamme variée de logiciels pour soutenir ses opérations. Ceux-ci incluent :
\begin{itemize}

  \item des outils de développement tels que Visual Studio code, Vim, IntelliJ IDEA et
    PyCharm pour la programmation et la création de solutions logicielles sur mesure;

  \item des logiciels de conception graphique tels que Figma, Adobe Creative Suite
    (Photoshop, Illustrator, InDesign) pour la création d’interfaces utilisateurs
    attrayantes et de graphiques de haute qualité;

  \item des outils de productivité tels que Microsoft Office (Word, Excel, PowerPoint)
    pour la gestion des documents et la communication professionnelle;

  \item des systèmes d’exploitation  (Windows, Linux, MacOS et FreeBSD), qui prennent en charge
    les différentes plates-formes et les besoins spécifiques des développeurs,
    des concepteurs et des administrateurs système;

  \item des outils de sécurité tels qu’Avast Antivirus Premium ainsi que des VPN comme ProtonVPN
    pour protéger les systèmes et les données contre les menaces en ligne.

\end{itemize}

\newpage


\section{Sujet}
\subsection{Contexte du Sujet}

À l'heure actuelle, l'étude de la généalogie est en pleine mutation grâce aux
avancées technologiques. Dans ce contexte évolutif, Mazala-Firm a identifié un
besoin crucial : celui d'une plateforme dédiée à la création collaborative et au
partage d'arbres généalogiques. Cette initiative répond à une demande croissante de
préservation et de partage de l’héritage familial à l’ère numérique. Elle
propose une solution innovante pour les utilisateurs qui souhaitent explorer et
partager leur lignée familiale.


La généalogie occupe une place particulière dans la société, offrant aux individus
une occasion unique de découvrir leurs racines et de se connecter à leur passé
familial. En facilitant l’accès à ces connaissances et en encourageant la
collaboration entre les membres de la famille, notre plateforme vise à renforcer
le lien intergénérationnel et à préserver l’histoire familiale pour les générations futures.

Les méthodes traditionnelles de création et de partage d’arbres généalogiques
sont souvent limitées par des contraintes physiques, comme la disponibilité de
documents papier ou la difficulté de les partager avec des membres de la famille
éloignés. En intégrant les dernières avancées technologiques, notre plateforme
offre une solution moderne et conviviale. Elle permet aux utilisateurs de créer,
de visualiser et de partager facilement leur arbre généalogique avec leur famille
et leurs proches, peu importe où ils se trouvent dans le monde.

Les objectifs spécifiques de notre projet incluent la facilitation de la collaboration
entre les membres de la famille, la convivialité de l’interface utilisateur et
l’intégration de fonctionnalités de partage avancées. En mettant l’accent sur
l’accessibilité et la simplicité d’utilisation, nous voulons démocratiser l’accès
à la généalogie et encourager un engagement actif dans la préservation de l’histoire familiale.

À long terme, notre plateforme pourrait jouer un rôle crucial dans la préservation
de l'histoire familiale pour les générations futures. Nous aspirons à développer
des fonctionnalités supplémentaires, telles que des analyses généalogiques avancées
et des outils de santé familiale, afin de créer une ressource complète pour les
utilisateurs soucieux de comprendre leur passé et de planifier leur avenir en
toute connaissance de cause.

\subsection{Problématique du sujet}
L’évolution technologique et la nécessité de préserver et de partager l’héritage
familial à l’ère numérique soulèvent plusieurs questions. Elles incluent :

\begin{itemize}
  \item Comment favoriser la collaboration pour la création collective d’arbres
    généalogiques parmi les membres de la famille, peu importe leur
    emplacement géographique?

  \item Comment rendre plus simple le partage d’informations généalogiques tout
    en assurant la protection de la vie privée et de la sécurité des données?

  \item Comment encourager la découverte de l’histoire familiale et inciter
    activement les gens à prendre soin de leur patrimoine familial?

  \item  Quelles sont les meilleures méthodes pour concevoir une plateforme web
    et mobile efficace pour la création collaborative et le partage d’arbres
    généalogiques?

  \item Quels sont les technologies et les outils les plus appropriés pour
    développer une plateforme de généalogie conviviale et sécuritaire?

  \item Quel choix architectural et quelle approche méthodologique adopter pour
    garantir la qualité, la performance et la sécurité de la plateforme?

\end{itemize}

\subsection{Intérêts du sujet}
La création collaborative et le partage d’arbres généalogiques sur une plateforme
web et mobile présentent un intérêt majeur à plusieurs niveaux.

D'abord, cette initiative répond à un besoin profond de notre société actuelle :
celui de préserver et de transmettre l’histoire familiale aux générations futures.
Elle permet aux individus d’explorer leurs racines et de découvrir leur lignée
familiale, ce qui contribue à renforcer le lien intergénérationnel et à promouvoir
un sentiment d’identité et d’appartenance au sein de la famille.

Ensuite, la dimension collaborative de cette plateforme offre aux utilisateurs une
occasion unique de partager leurs connaissances et leurs découvertes avec leur famille
élargie, favorisant ainsi les échanges interculturels et intergénérationnels.
Cette collaboration peut également mener à la découverte de nouvelles branches
familiales et à la réconciliation de membres de la famille séparés par le temps ou la distance.

Sur le plan technologique, ce projet représente une avancée significative dans le
domaine de la généalogie numérique. Il intègre les dernières innovations en
matière de conception d’interfaces utilisateurs conviviales et d’analyse de données.
En proposant une expérience utilisateur intuitive et personnalisée, il vise à
démocratiser l’accès à la généalogie et à encourager un engagement actif dans
la préservation de l’histoire familiale.

Enfin, sur le plan social, cette plateforme peut contribuer à sensibiliser les
individus aux enjeux liés à la santé familiale. Elle permet par exemple d’identifier
les prédispositions génétiques à certaines maladies et d’adopter des comportements
préventifs, mais aussi la consanguinité. En favorisant la prise de conscience des
liens familiaux et des risques pour la santé, cette plateforme joue un rôle crucial
dans la promotion du bien-être familial et de la solidarité intergénérationnelle.

Dans l’ensemble, la création d’une plateforme web et mobile dédiée à la généalogie
collaborative et au partage d’arbres généalogiques constitue une occasion unique
de combiner les avancées technologiques aux besoins fondamentaux de préservation
de l’histoire familiale et de renforcement du lien familial. En mettant l’accent
sur l’accessibilité, la convivialité et la sécurité des données, ce projet vise à
devenir une référence incontournable dans le domaine de la généalogie numérique.
Il contribue ainsi à enrichir le patrimoine familial et culturel de notre société.

\section{Concepts liés au sujet}
Le sujet que nous abordons englobe une série de concepts clés qui sont au cœur
de notre projet de création collaborative et de partage d’arbres généalogiques.
Ces concepts incluent :
\begin{itemize}
  \item \textbf{Généalogie :} il s’agit de l’étude des relations de parenté et de la
    succession des générations au sein d’une famille. La généalogie est le
    fondement même de notre projet, offrant aux individus la possibilité de
    découvrir et de préserver leur héritage familial.

  \item \textbf{Arbre généalogique :} un arbre généalogique est une représentation graphique
    des liens de parenté entre les membres d’une famille. Il permet de visualiser
    les relations familiales sur plusieurs générations et de retracer l’histoire
    familiale de manière structurée et organisée.

  \item \textbf{Collaboration en ligne :} Ce terme désigne le processus de travail d’équipe
    visant à atteindre un objectif commun par le biais de plateformes numériques.
    Dans notre cas, la collaboration en ligne revêt une importance particulière,
    car elle permet aux membres de la famille de contribuer à la construction et
    à l’enrichissement des arbres généalogiques, peu importe leur emplacement géographique.

  \item \textbf{Plateformes web et mobiles :} nous envisageons de développer une plateforme
    accessible à la fois sur le web et sur les appareils mobiles, offrant ainsi
    aux utilisateurs la flexibilité nécessaire pour accéder à leurs données
    familiales où qu’ils se trouvent. Cette approche garantit une expérience
    utilisateur optimale et facilite le partage d’informations entre les membres de la famille.

  \item \textbf{Sécurité des données :} La protection des informations personnelles des
    utilisateurs et des données sensibles relatives à l’histoire familiale est
    une priorité pour notre projet. Nous mettrons en place des mesures de sécurité
    robustes pour garantir la confidentialité et l’intégrité des données, tout en
    offrant aux utilisateurs un contrôle total sur leurs informations.

  \item \textbf{Partage de données :} nous accordons une grande importance au partage
    efficace et sécuritaire des données entre les membres de la famille. Notre
    plateforme facilitera ce processus en permettant aux utilisateurs d’échanger
    et de transmettre des informations de manière transparente, tout en respectant
    les préférences de confidentialité de chacun.

  \item \textbf{Analyse des données :} en plus de permettre la création et le partage
    d’arbres généalogiques, notre plateforme offrira des outils d’analyse avancés
    pour que les utilisateurs puissent explorer et comprendre leur histoire familiale
    de manière approfondie. Des fonctionnalités telles que des statistiques
    démographiques, des tendances généalogiques et des analyses de santé familiale
    seront intégrées pour fournir des informations précieuses aux utilisateurs et
    les aider à mieux comprendre leur lignée et leur héritage.

  \item \textbf{Accessibilité :} enfin, nous nous engageons à rendre notre plateforme
    accessible à tous, y compris aux personnes ayant des besoins spécifiques en
    matière d’accessibilité. Nous veillerons à ce que l’interface utilisateur soit
    intuitive et facile à utiliser pour que chacun puisse profiter pleinement de
    l’expérience généalogique, quel que soit son niveau de compétences technologiques.

\end{itemize}

Ces concepts joueront un rôle essentiel dans la conception et la mise en œuvre
de notre plateforme. Ils guideront nos choix technologiques et nos stratégies de
développement. En intégrant ces principes fondamentaux, nous espérons créer une
solution innovante et inclusive qui répond aux besoins variés des utilisateurs
tout en préservant l’histoire familiale pour les générations futures.

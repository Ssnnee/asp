\documentclass[12pt,french]{extreport}

\usepackage[backend=biber]{biblatex}
\addbibresource{bibliographie.bib}

\usepackage{titlesec}
\usepackage[french]{babel}
\usepackage{fancyhdr}
\usepackage{dirtytalk}
% \usepackage[fontsize=12pt]{scrextend}
% 8pt, 9pt, 11pt, 12pt, 14pt, 17pt, and 20pt.

\usepackage{graphicx}
\usepackage{geometry}
\usepackage{float}

\usepackage{tabularx}
\usepackage{diagbox}
\usepackage{multirow}

\usepackage{hyperref}

\usepackage[edges]{forest}
\usetikzlibrary{fit}
\usepackage{tikz}
\usepackage{color}
\usepackage{tikz-uml}
\usepackage{pgf-umlsd}
\usetikzlibrary{positioning}
\usepackage{pgfgantt}
\usepackage{rotating}

\usepackage{amssymb}
\usepackage{pifont}
\usepackage{fancybox}
\usepackage{lmodern}
\usepackage{tcolorbox}

\usetikzlibrary{%
  arrows,%
  calc,
  shapes,
  arrows,
  shapes.misc,
  shapes.arrows,%
  chains,%
  matrix,%
  positioning,%
  scopes,%
  decorations.pathmorphing,%
  shadows%
}
\usepackage{amsmath}
\usepackage{relsize}

\newcommand{\cmark}{\textcolor{green}{\ding{51}}} % check mark
\newcommand{\xmark}{\textcolor{red}{\ding{55}}} % cross mark

% \newcommand{\greencheck}{\textcolor{green}{\ding{51}}} % Green check mark
% \newcommand{\redcheck}{\textcolor{red}{\ding{55}}}   % Red cross mark

%Used to draw gantt charts, which I will use for the calendar.
%Let's define some awesome new ganttchart elements:
\newganttchartelement{orangebar}{
    orangebar/.style={
        inner sep=0pt,
        draw=red!66!black,
        very thick,
        top color=white,
        bottom color=orange!80
    },
    orangebar label font=\slshape,
    orangebar left shift=.1,
    orangebar right shift=-.1
}

\newganttchartelement{bluebar}{
    bluebar/.style={
        inner sep=0pt,
        draw=purple!44!black,
        very thick,
        top color=white,
        bottom color=blue!80
    },
    bluebar label font=\slshape,
    bluebar left shift=.1,
    bluebar right shift=-.1
}

\newganttchartelement{greenbar}{
    greenbar/.style={
        inner sep=0pt,
        draw=green!50!black,
        very thick,
        top color=white,
        bottom color=green!80
    },
    greenbar label font=\slshape,
    greenbar left shift=.1,
    greenbar right shift=-.1
}

\usepackage{pgfgantt}

\usetikzlibrary{shapes.geometric, arrows, trees, positioning, fit, calc}

% \usepackage{biblatex}
% \addbibresource{bibliographie.bib}

\setcounter{secnumdepth}{3}
\setcounter{tocdepth}{3}
\makeatletter\@addtoreset{chapter}{part}\makeatother


\usepackage[nolist]{acronym}
% \usepackage{acronym}


\geometry{
 a4paper,
 total={170mm,257mm},
 left=20mm,
 top=20mm,
 }

\newcommand{\projettheme}{
  Étude et mise en œuvre d'une plateforme web et mobile pour la création
  collaborative et le partage d'arbres généalogiques à Mazala-Firm
}
\newcommand{\projetauthor}{
  M. Samuel Exaucé NANDI \\
  M. Dieu-veille Frédy ONIANGUE-DESO
}

\newcommand{\firm}{
  Mazala-Firm
}


% \title{\projettheme}
% \date{Mai 2024}


\fancyhf{}
\chead{\projettheme}
\lfoot{LPGL: Samuel Exaucé NANDI \& Dieu-veille Frédy ONIANGUE-DESO \qquad  2023-2024}
\rfoot{\number\value{page}}

\renewcommand{\headrulewidth}{0.4pt}
\renewcommand{\footrulewidth}{0.4pt}
\pagestyle{fancy}

\newcommand{\mazf}{
  Mazala-Firm
}

\fancypagestyle{plain}{
  \fancyhf{} % Clear all header and footer fields
  \chead{\projettheme}
  \lfoot{LPGL: Samuel Exaucé NANDI \& Dieu-veille Frédy ONIANGUE-DESO \qquad  2023-2024}
  \rfoot{\thepage}
  \renewcommand{\headrulewidth}{0.4pt}
  \renewcommand{\footrulewidth}{0.4pt}
}

\begin{document}

\pagenumbering{roman}

\begin{titlepage}
  \begin{center}

    \large {
      \textbf{
        E. S. G. A. E \\
        ÉOLE SUPÉRIEURE DE GESTION \\
        ET D’ADMINISTRATION DES ENTREPRISES \\
      }
    }

    \vspace{0.2cm}

    \large {
      \textbf{ Brazzaville – Congo } \\
      Agrément définitif par Arrêté n°10403/MESRSIT/CAB du 25 Août 2023 \\
      Accrédité par le Conseil Africain et Malgache pour l’Enseignement Supérieur (CAMES) \\
      B.P: 2339 Tel. (+242) 06 691 96 79 / 05 739 26 89 \\
      E-mail: \href{mailto:esgae@yahoo.fr}{esgae@yahoo.fr} ; \href{mailto:esgae@esgae.org}{esgae@esgae.org} \\
      Site Web: \url {https://www.esgae.org} \\
    }

    \vspace{0.5cm}

    \dotfill

    \vspace{0.5cm}

    \large {
      \textbf { \underline {PROJET INFORMATIQUE} }
    }

    \vspace{0.2cm}
    \large {
      \textbf{POUR L’OBTENTION DU DIPLOME DE LICENCE PROFESSIONNELLE}
    }

    \vspace{0.2cm}
    \large {
      \textbf{PARCOURS : GENIE LOGICIEL}
    }

    \vspace{1.5cm}

    \normalsize {
    \textbf { \underline {THÈME} }
    }

    \begin{tcolorbox}[sharp corners=uphill,
    arc=6mm,boxrule=2mm,boxsep=5mm,
    ]
    \centering
    \large {
      ÉTUDE ET MISE EN ŒUVRE D’UNE PLATEFORME WEB ET MOBILE POUR LA CRÉATION
      COLLABORATIVE ET LE PARTAGE D’ARBRES GÉNÉALOGIQUES À MAZALA-FIRM
    }
    \end{tcolorbox}
  \end{center}

  \vspace{2.0cm}

  \noindent
  \begin{minipage}[t]{0.45\textwidth}
    \large{
      \textbf{
        Présenté et soutenu par : \\
      }
    }
    \normalsize{
      \projetauthor
    }

  \end{minipage}
  \hfill
  \begin{minipage}[t]{0.45\textwidth}
    \raggedleft
    \large{
      \textbf{
        Sous la direction de : \\
      }
      M. Chistopher BANDZOUZI \\
      Ingénieur informaticien
    }
  \end{minipage}

  \vspace{3.9cm}
  \begin{center}
    \large {
      \textbf { \underline {Année académique : 2023 – 2024} }
    }

  \end{center}


\end{titlepage}


\chapter*{Dédicace}
\addcontentsline{toc}{chapter}{Dédicace}
Je dédie ce modeste travail à :

\begin{itemize}
  \item {
      À mes parents ONIANGUE-DESO Rock Frédy et NGANBOMI Zita, pour leur amour
      inconditionnel, leur soutien constant et leurs sacrifices sans fin. Vos
      encouragements m’ont donné la force de poursuivre mes études et de réaliser ce mémoire.
    }
  \item {
      À mes professeurs, pour leur expertise, leur patience et leur inspiration.
      Leurs conseils éclairés ont enrichi mon parcours académique et ont façonné ma pensée critique.

    }
  \item {
      À mes amis, pour leur soutien indéfectible et leurs encouragements tout au
      long de cette aventure. Leurs encouragements m’ont permis de surmonter les défis avec confiance.

    }

  \item {
      À mon adorable conjointe DOUMA LEBONGUI Merveille Chrisnat.

    }

  \item {
      Enfin, à tous ceux qui ont croisé mon chemin et m’ont apporté leur aide,
      je vous suis profondément reconnaissant. Ce mémoire est le fruit de nos
      efforts collectifs et je vous en suis éternellement reconnaissant.
    }

    \vspace{0.2cm}
    \begin{flushright}
      \large {
        \textbf {
          Dieu-veille Frédy ONIANGUE-DESO
        }
      }
    \end{flushright}


    \newpage
\chapter*{Dédicace}

    \vspace{2cm}
    Je dédie ce travail à mon cousin  Inefable KOUMBA, de qui je tiens ce 	projet (à quelques personnalisation près).
    \vspace{0.2cm}
    \begin{flushright}
      \large {
        \textbf {
          Samuel Exaucé NANDI
        }
      }
    \end{flushright}

\end{itemize}


\chapter*{Remerciements}
\addcontentsline{toc}{chapter}{Remerciements}

Nous souhaitons exprimer nos plus sincères remerciements à :

\begin{itemize}
  \item Monsieur le professeur titulaire Roger Armand MAKANY, Directeur Général de l’ESGAE,
      pour nous avoir accueillis dans cette enceinte universitaire;

  \item Monsieur Christopher BANDZOUZI notre Directeur de Projet pour sa
      longue patience et sa compréhension sans égale;

  \item Mesdames Vérita YAOUE-NGUESSIMOU et Marly POUABOUD-EUGE, étudiantes à l’ESGAE, pour
      leur aide précieuse;

  \item Tout le personnel enseignant et administratif de l’ESGAE;


  \item Tout le personnel de MAZALA-FIRM pour sa disponibilité, son aide et son
      accueil chaleureux;

  \item Tous nos camarades de classe et nos collègues de l’école.

  \item Nos parents et nos familles pour leur soutien.

\end{itemize}


\listoffigures
\addcontentsline{toc}{chapter}{\listfigurename}
\listoftables
\addcontentsline{toc}{chapter}{\listtablename}
% Ajouter le sommaire%

\chapter*{Sommaire}
\addcontentsline{toc}{chapter}{Sommaire}
\noindent\begin{tabularx}{\textwidth}{Xr}
  \textbf{Introduction} \dotfill \pageref{chap:introduction} \\

  \textbf{Première Partie : Présentation générale de Mazala-Firm} \dotfill \pageref{part:presentation} \\

  \textbf{Deuxième Partie : Analyse et conception} \dotfill \pageref{part:analyse-et-conception} \\

  \textbf{Troisième Partie : Evaluation et Réalisation} \dotfill \pageref{part:evaluation-et-realisation} \\

  \textbf{Conclusion} \dotfill \pageref{chap:conclusion} \\

  \textbf{Table des matières} \dotfill \pageref{sec:tableofcontents} \\

  \textbf{Bibliographie} \dotfill \pageref{sec:bibliographie} \\

  \textbf{Webographie} \dotfill \pageref{sec:webographie} \\
\end{tabularx}

\chapter*{Abréviation et Sigles}
% \thispagestyle{nohede
\begin{acronyms}
  \acro{E.S.G.A.E}{École Supérieure de Gestion et d’Administration des Entreprises}

\end{acronyms}



\pagenumbering{arabic}
\chapter*{Introduction}
Aujourd’hui, les technologies informatiques influencent considérablement notre
vie quotidienne. La convergence des plateformes web et mobiles a remodelé la
façon dont nous nous connectons, partageons et préservons nos informations.
En tant que discipline, la généalogie est ancrée dans la préservation et la
transmission de l’héritage familial. Avec l’avènement des technologies modernes,
il devient impératif d’explorer de nouvelles perspectives pour faciliter la
création et le partage d’arbres généalogiques. Dans le cadre de notre projet pour
l’obtention de la licence professionnelle en filière informatique de gestion,
option génie logiciel, au sein de l'\ac{E.S.G.A.E} , nous nous sommes engagés dans l’étude et la mise
en œuvre d’une plateforme visant à simplifier et à enrichir l’expérience de
reconstitution et de partage d’arbres généalogiques.

Nous avons choisi comme thème de projet \say{\projettheme}. Cette initiative vise à répondre à un besoin croissant : la préservation
et le partage de l’héritage familial à l’ère numérique.
Notre projet s’articule autour de trois objectifs principaux :
1. Faciliter la création collaborative d’arbres généalogiques : nous souhaitons
fournir une plateforme conviviale où les membres de la famille peuvent collaborer
efficacement pour construire et partager leurs arbres généalogiques.
2. Simplifier le partage des informations généalogiques : nous souhaitons créer
une interface intuitive sur les plateformes web et mobiles, permettant aux utilisateurs
de partager facilement leurs arbres généalogiques avec leurs proches.
3. Promouvoir la découverte de son histoire ou de son héritage familial.
Notre plateforme est destinée à tous ceux qui souhaitent explorer et partager
leur lignée familiale, que ce soit des novices ou des professionnels de la généalogie.
Pour mener à bien ce projet, nous avons opté pour une approche méthodologique agile :
la méthode agile \ac{2TUP}.
Cette méthode, axée sur la collaboration et l’itération, nous permettra de développer
la plateforme de manière incrémentale, en nous adaptant aux besoins changeants des utilisateurs.

Tout au long du processus, nous mettrons en œuvre les meilleures pratiques de
développement logiciel pour garantir la qualité, la sécurité et la performance
de notre solution. Notre objectif est de fournir une expérience enrichissante à
toute personne désireuse de découvrir l’histoire de sa famille et de la partager avec ses proches.


\newpage

\chapter{La Structure d’Accueil et le Sujet}
\section{La Structure d’Accueil}
\subsection{Historique}

\mazf est une agence de développement spécialisée dans la création de solutions
numériques novatrices. Créée en 2021 par Inefable KOUMBA, un passionné des nouvelles
technologies, \mazf s'est rapidement établie comme un acteur de premier plan dans ce domaine.
L'histoire de \mazf débute avec Inefable KOUMBA, un ingénieur informatique
visionnaire qui avait pour but de fonder une agence proposant des services complets
en développement web. Fort de son expertise et de sa passion pour l’innovation,
il rassemble une équipe de talents aux compétences variées, allant de l’analyse
de données au design UX/UI en passant par le développement logiciel.
Dès ses débuts, \mazf se distingue par son approche holistique du développement
web. L’agence comprend rapidement que la clé du succès réside dans la capacité à
transformer les données en informations exploitables. Ainsi, l’analyse des données
devient l’un des piliers fondamentaux de l’activité de l’agence. Grâce à une
expertise avancée en analyse de données, \mazf aide ses clients à tirer
parti de leurs données pour prendre des décisions éclairées et stratégiques.
Parallèlement, l’agence met un point d’honneur à offrir des expériences utilisateur
exceptionnelles à travers ses services de conception UI/UX. En combinant esthétique
et fonctionnalité, les designers de \mazf façonnent des interfaces intuitives
et attractives, garantissant ainsi des interactions mémorables avec les produits
numériques de ses clients. Le développement logiciel sur mesure est une autre
spécialité de \mazf. Que ce soit pour la création d’applications mobiles,
de plateformes web complexes ou de logiciels d’entreprise, l’équipe de développeurs
de l’agence excelle dans l’art de concevoir des solutions sur mesure répondant
parfaitement aux besoins spécifiques de chaque client.

\subsection{Missions}
Les missions de \firm sont mutliples et s’articulent autour de plusieurs axes. L’entreprise
est engagée dans la transformation de données en informations exploitables, dans
la création d’expériences utilisateur simples et intuitives, dans le développement de solutions
logicielles personnalisées, dans la fourniture de conseils stratégiques et d’audits
informatiques pertinents, dans la gestion fiable des systèmes et réseaux informatiques,
ainsi que dans la proposition de solutions de stockage sur le nuage sécurisées et flexibles.

Voici les missions de \firm :
\begin{enumerate}
  \item Analyse des données : Notre mission est de transformer les données brutes
    en informations exploitables. Nous utilisons une expertise avancée en analyse
    de données pour aider nos clients à comprendre, interpréter et exploiter efficacement
    les données pour prendre des décisions stratégiques et éclairées.
  \item UI \& UX Design : nous nous engageons à créer des expériences utilisateur mémorables
    en combinant l’esthétique et la fonctionnalité. Notre équipe de concepteurs
    d’interface utilisateur/expérience utilisateur travaille en étroite collaboration
    avec nos clients pour concevoir des interfaces intuitives, attrayantes et ergonomiques,
    garantissant ainsi une satisfaction utilisateur optimale.
  \item Développement logiciel : notre objectif est de créer des solutions sur
    mesure qui répondent parfaitement aux besoins spécifiques de chaque client.
    Que ce soit pour le développement d’applications mobiles, de plateformes Web
    ou de logiciels d’entreprise, notre équipe de développeurs s’engage à fournir
    des solutions innovantes et de haute qualité.
  \item Conseil et audit informatique : nous proposons des conseils stratégiques
    et des audits informatiques afin d’aider nos clients à optimiser leurs systèmes
    et leurs investissements technologiques. Notre mission est d’accompagner nos
    clients dans leur transformation digitale en leur fournissant des recommandations
    avisées et des solutions adaptées à leurs besoins spécifiques.
  \item Administration des systèmes et des réseaux informatiques : nous nous
    engageons à assurer une gestion efficace et optimale des systèmes et des
    réseaux informatiques de nos clients. Notre équipe d’administrateurs système
    expérimentés travaille en coulisses pour garantir la fiabilité, la sécurité et
    la performance des infrastructures informatiques de nos clients.
  \item Stockage des données sur le nuage : Notre mission est de fournir des solutions
    de stockage sur le nuage sécurisées et flexibles. En collaboration avec les
    principaux fournisseurs de services infonuagiques, nous offrons à nos clients
    un accès simple et sécurisé à leurs données, où qu’ils se trouvent.
\end{enumerate}

\subsection{Organigramme Général}
L’organigramme de \firm est bien structuré et se présente comme suit :






\subsection{Attributions des structures}
L’organigramme ci-dessus illustre le rôle crucial que jouent chaque poste et
département au sein de \firm. Voici une brève description de chaque
structure et de ses principales responsabilités :
\begin{enumerate}
  \item Directeur général : Chef d’entreprise, responsable de la gestion globale,
    des décisions stratégiques et de la supervision des départements et des employés.

  \item Secrétaire : assiste les dirigeants et les cadres supérieurs dans leurs
    tâches administratives, notamment la gestion du courrier et des réunions.

  \item Responsable comptable : gère les finances de l’entreprise, y compris la
    tenue des livres comptables et la préparation des rapports financiers.

  \item Responsable marketing : planifie, mets en œuvre et gère les stratégies
    marketing, y compris la publicité et les relations publiques.

  \item Responsable technique : supervise les aspects techniques de l’entreprise,
    notamment l’infrastructure informatique et les processus de développement.

  \item Community manager : gère et anime les communautés en ligne, interagis
    avec les clients et crée du contenu pour engager la communauté.

  \item Équipe de développement : conçois, développe et maintiens les logiciels
    et les applications de l’entreprise et des produits.

  \item Équipe designer : responsable de la conception visuelle des produits,
    incluant la conception graphique et l’interface utilisateur.

\end{enumerate}

\subsection{Situation informatique}
\firm dispose d’un département informatique dédié, supervisé par le Responsable
technique. Ce département est structuré en deux équipes principales :
\begin{itemize}

  \item Équipe de développement: conçois et développe des solutions logicielles
    personnalisées en collaboration avec les clients.
  \item Équipe de design: crée des interfaces utilisateur attrayantes et intuitives
    pour les produits numériques de l’entreprise.
\end{itemize}

\subsubsection{Personnel informatique}
L’équipe du service informatique comprend une équipe talentueuse composée de :
\begin{itemize}
  \item un chef de Service technique
  \item des développeurs logiciels
  \item des designers UX/UI
  \item des ingénieurs et techniciens informatiques spécialisés dans
  \item le développement et la maintenance des solutions logicielles et des interfaces utilisateur.
\end{itemize}


Cette structure organisationnelle garantit que \firm dispose des ressources
et des compétences nécessaires pour répondre aux besoins technologiques de ses
clients et fournir des solutions informatiques de haute qualité et innovantes.


\subsubsection { Matériels informatiques }
L’efficacité de \firm dépend étroitement de notre matériel informatique.
En tant qu’entreprise spécialisée dans le développement web et le design, nous
comprenons l’importance d’un équipement de qualité pour soutenir nos équipes dans
la réalisation de projets innovants. Ci-dessous, nous présentons les principaux
composants de notre infrastructure informatique, démontrant ainsi notre engagement
envers des solutions performantes et une satisfaction client optimale.
\begin{itemize}
  \item Ordinateurs portables : des ordinateurs portables hautes performances
    équipés de processeurs rapides, de suffisamment de mémoire RAM et de stockage SSD
    pour permettre aux développeurs et aux designers de travailler efficacement, que ce soit au bureau ou en déplacement.
  \item Des stations de travail puissantes pour les tâches de développement et de design
    les plus exigeantes, avec des processeurs multicœurs, une mémoire RAM étendue
    et des cartes graphiques dédiées pour accélérer les rendus et la visualisation.
  \item Moniteurs haute résolution : des moniteurs haute résolution avec des couleurs
    précises et un grand espace de travail pour les designers afin qu’ils puissent visualiser leurs créations dans les moindres détails.
  \item Tablettes graphiques : Des tablettes graphiques professionnelles pour les
    designers, permettant de dessiner et de retoucher des images avec précision,
    offrant une expérience de dessin naturelle et intuitive.
  \item Serveurs : des serveurs hautes performances pour héberger les applications
    Web et les bases de données de l’entreprise, assurant ainsi une disponibilité
    et une sécurité maximales pour les services en ligne de \firm.
  \item Équipements réseau : Des équipements réseau tels que des routeurs, des
    commutateurs Ethernet et des points d’accès Wi-Fi sont nécessaires pour
    assurer une connectivité réseau fiable et sécuriser dans tout le bureau.
  \item Accessoires : claviers ergonomiques, souris de précision, casques audio et
    autres accessoires informatiques pour permettre à chaque membre de l’équipe
    de travailler confortablement et efficacement.

\end{itemize}


\subsubsection   { Logiciels informatiques }
\firm utilise une gamme variée de logiciels pour soutenir ses opérations. Ceux-ci incluent :
\begin{itemize}

  \item des outils de développement tels que Visual Studio code, Vim, IntelliJ IDEA et
    PyCharm pour la programmation et la création de solutions logicielles sur mesure.
  \item Des logiciels de conception graphique tels que Figma, Adobe Creative Suite
    (Photoshop, Illustrator, InDesign) pour la création d’interfaces utilisateurs
    attrayantes et de graphiques de haute qualité.
  \item Des outils de productivité tels que Microsoft Office (Word, Excel, PowerPoint)
    pour la gestion des documents et la communication professionnelle.
  \item Des systèmes d’exploitation comme Windows et Linux, qui prennent en charge
    les différentes plates-formes et les besoins spécifiques des développeurs.
  \item Des outils de sécurité tels qu’Avast Antivirus Premium pour protéger les
    systèmes et les données contre les menaces en ligne.
\end{itemize}

\newpage




\part{Analyse et Conception }
\chapter{Etude préliminaire et Méthode}

Dans ce chapitre, nous allons présenter l’étude préliminaire de notre projet de
création collaborative et de partage d’arbres généalogiques, ainsi que la
méthodologie que nous allons suivre pour sa conception et son développement.
Nous commencerons par une analyse approfondie de l’existant. Nous décrirons la
situation actuelle de \firm dans le domaine de la généalogie en ligne, nous
identifierons ses limites et nous proposerons des solutions pour les surmonter.

Ensuite, nous présenterons la méthode que nous allons utiliser pour concevoir notre
plateforme. Nous mettrons l’accent sur le langage de modélisation \ac{UML} ,
ainsi que sur les différents types de diagrammes UML que nous utiliserons pour
représenter notre système de manière visuelle et structurée. Nous discuterons
également des processus de développement que nous allons adopter, notamment
le \ac{UP} et le \ac{2TUP}. Nous verrons comment ils peuvent être intégrés
à UML pour assurer une gestion efficace du projet.

Ce chapitre posera les bases nécessaires à la réalisation de notre projet en
fournissant une compréhension approfondie de son contexte et en établissant
les lignes directrices pour sa conception et son développement. En combinant
une analyse approfondie de l’existant avec une méthodologie de conception
rigoureuse, nous sommes bien placés pour créer une plateforme généalogique
collaborative et innovante qui répondra aux besoins de nos utilisateurs tout
en garantissant sa robustesse et sa pérennité.

\section{Etude de l'existant}
\subsection{Description des activités (la situation actuelle)}
Actuellement, \firm ne dispose pas d’une plateforme dédiée à la création
collaborative et au partage d’arbres généalogiques. Les activités liées à la
généalogie sont principalement effectuées de manière traditionnelle, impliquant
la collecte de documents papier, la communication verbale et l’échange de fichiers
numériques par courrier électronique ou d’autres moyens non structurés. Les
membres de la famille rencontrent souvent des difficultés pour collaborer
efficacement à la création et au partage d'arbres généalogiques en raison de
l'absence d'un outil centralisé et convivial.

\subsection{Critique de l'existant (les limites)}
Cette approche traditionnelle présente plusieurs limites. Elle rend difficile
la collaboration entre les membres de la famille, en particulier ceux qui sont
éloignés géographiquement, et ne permet pas un partage facile et sécurisé des
informations généalogiques. De plus, la préservation à long terme de l’histoire
familiale est compromise, car les documents papier peuvent se perdre ou se
détériorer avec le temps, et les fichiers numériques peuvent être dispersés et mal organisés.

\subsection{Proposition de solutions}
Pour surmonter ces défis, nous proposons les solutions suivantes :
\begin{itemize}
  \item  Utiliser des logiciels de généalogie hors ligne : cette solution implique
    l’installation de logiciels sur des ordinateurs personnels pour créer et gérer
    des arbres généalogiques. Cependant, cela peut présenter certaines limites
    en termes de collaboration et de partage, car les données sont souvent stockées
    localement et ne permettent pas une collaboration facile entre les membres de la famille.

  \item  Utiliser des sites Web de généalogie : cela consiste à
    utiliser des sites Web spécialisés en généalogie pour créer et partager des arbres
    généalogiques. Cette méthode est pratique, car elle permet de partager ses arbres
    avec d’autres utilisateurs. Cependant, les fonctionnalités des sites sont limitées et ne
    permettent pas toujours une collaboration efficace.

  \item Développer d’une plateforme web et mobile dédiée à la création
    collaborative et au partage d’arbres généalogiques : Cette solution propose
    de créer une plateforme personnalisée qui répond spécifiquement aux besoins de
    \firm Elle permettra aux utilisateurs de collaborer à la création d’arbres
    généalogiques, de partager facilement des informations avec leur famille et
    leurs proches, et offrira des fonctionnalités avancées, notamment la gestion
    de la confidentialité des données et des outils d’analyse.

\end{itemize}

\subsection{Choix de la solution}
Parmi les solutions présentées, nous choisissons de développer une plateforme web
et mobile dédiée à la création collaborative et au partage d’arbres généalogiques.
Elle présente la meilleure combinaison de convivialité, de fonctionnalités avancées
et de contrôle sur la confidentialité des données. En offrant une plateforme
centralisée et sécurisée, nous pouvons garantir une expérience utilisateur optimale
tout en répondant aux besoins diversifiés des utilisateurs.

En résumé, notre choix reflète notre engagement à fournir une solution
efficace et innovante pour relever les défis de \firm.



\part{Evaluation et Réalisation}
\label{part:evaluation-et-realisation}
\chapter{Evaluation du projet}
Ce chapitre traite de l’organisation et de la gestion du projet, en analysant
les éléments clés nécessaires à une évaluation efficace.

\section{Organisation du projet}
L’organisation du projet implique la définition des rôles, des responsabilités
et des processus de travail. Une structure organisationnelle claire facilite
la coordination et la communication entre les membres de l’équipe, assurant
ainsi une progression harmonieuse du projet.

% Dans le cadre de ce projet, l’organisation est structurée autour de deux
% principaux processus : le processus de développement et le processus de
% validation.
%
% \begin{table}[htbp]
%   \centering
%   \begin{tabularx}{\textwidth}{|l|X|}
%     \hline
%     \textbf{Processus} & \textbf{Phases} \\ \hline
%     \multirow{4}{*}{Développement} & Analyse des besoins : Définir les fonctionnalités nécessaires à la création collaborative et au partage d’arbres généalogiques. \\
%      & Conception : Concevoir l'architecture logicielle de la solution, ainsi que l'interface utilisateur pour le web et le mobile. \\
%      & Développement : Implémenter les fonctionnalités selon les spécifications définies lors de l'analyse et de la conception. \\
%      & Tests : Réaliser des tests unitaires et d'intégration pour garantir le bon fonctionnement de l'application. \\ \hline
%     \multirow{2}{*}{Validation} & Vérification : Vérifier que les fonctionnalités implémentées répondent aux besoins et aux spécifications du projet. \\
%      & Validation : Valider l'application avec les utilisateurs finaux pour recueillir leurs retours et effectuer les ajustements nécessaires. \\ \hline
%   \end{tabularx}
%   \caption{Organisation du projet}
% \end{table}

\section{Intervenant}

\begin{table}[htbp]
  \centering
  \begin{tabularx}{\textwidth}{|l|l|X|}
    \hline
    \textbf{Intervenants} & \textbf{Fonctions} & \textbf{Rôles} \\ \hline
    M. Christopher BANDZOUZI & Ingénieur Informaticien & Directeur de projet  \\ \hline
    M. Samuel Exaucé NANDI & Etudiant & Réalisateur \\ \hline
    M. Dieu-Veille Frédy ONIANGUE-DESO & Etudiant & Réalisateur \\ \hline
    Me. Rovélia MOUNTOU & Sécrétaire Mazala-Firm & Tutrice de stage \\ \hline
  \end{tabularx}
  \caption{Intervenants}
\end{table}

\section{Planification des tâches}
La planification des tâches consiste à décomposer le projet en activités
distinctes et à établir un calendrier pour leur réalisation. Elle permet de
suivre les progrès, de gérer les ressources efficacement et de
respecter les délais.

\begin{figure}[H]
  \centering
  % \begin{tabularx}{\textwidth}{|l|l|X|}
  %   \hline
  %   \textbf{Processus} & \textbf{Phases} & \textbf{Tâches} \\ \hline
  %   \multirow{2}{*}{Développement} & Analyse des besoins & - Étude des fonctionnalités nécessaires à la création collaborative et au partage d'arbres généalogiques. \\
  %    &  & - Identification des exigences de performance, de sécurité et de convivialité de la plateforme. \\ \cline{2-3}
  %   & Conception & - Conception de l'architecture logicielle de la solution, en mettant l'accent sur la scalabilité et la maintenabilité. \\
  %   &  & - Conception de l'interface utilisateur pour le web et le mobile, en tenant compte des principes de conception UX/UI. \\ \cline{2-3}
  %   & Développement & - Implémentation des fonctionnalités de création collaborative d'arbres généalogiques sur la plateforme web. \\
  %   &  & - Développement des fonctionnalités de partage et de visualisation des arbres généalogiques sur les applications mobiles. \\ \cline{2-3}
  %   & Tests & - Réalisation de tests unitaires et d'intégration pour garantir le bon fonctionnement des fonctionnalités développées. \\
  %   &  & - Tests de performance pour évaluer la réactivité et la scalabilité de la plateforme. \\ \hline
  %   \multirow{2}{*}{Déploiement} & Mise en production & - Configuration des serveurs et déploiement de l'application web Next.js sur un environnement de production sécurisé. \\
  %   &  & - Publication des applications mobiles sur les stores (App Store et Google Play) après une phase de tests approfondis. \\ \cline{2-3}
  %   & Formation & - Formation des utilisateurs finaux à l'utilisation de la plateforme, en mettant l'accent sur les fonctionnalités clés et les bonnes pratiques. \\
  %   &  & - Documentation complète de la plateforme pour une référence ultérieure. \\ \hline
  %   \multirow{2}{*}{Suivi} & Maintenance & - Surveillance continue de la plateforme pour détecter et corriger les éventuels problèmes de performance ou de sécurité. \\
  %   &  & - Mise à jour régulière de la plateforme avec de nouvelles fonctionnalités et correctifs de bugs. \\ \hline
  % \end{tabularx}
  % \caption{Planification des tâches}
  \includegraphics[width=1\textwidth]{capture/task.png}
  \caption{Planification des tâches}
\end{figure}


\section{Diagramme de Gantt}
Le diagramme de Gantt est un outil visuel de gestion de projet qui affiche les
tâches à accomplir sur une ligne de temps. Il permet de suivre les progrès,
d’identifier les dépendances entre les tâches et de prévoir les éventuels retards.

\newpage
% \begin{landscape}
%  \begin{sidewaysfigure}
%   \begin{ganttchart}[
%     hgrid,
%     vgrid,
%     x unit=0.5cm,
%     y unit title=0.7cm, % Augmenter légèrement la hauteur de chaque unité en titre
%     y unit chart=0.7cm, % Augmenter légèrement la hauteur de chaque unité en charte
%     time slot format=isodate,
%     title/.append style={draw=none, fill=gray!30},
%     title label font=\sffamily\bfseries\footnotesize,
%     title height=1,
%     title label anchor/.style={below=-1.6ex},
%     include title in canvas=false,
%     bar label font=\small,
%     bar label node/.append style={align=left},
%     bar/.append style={draw=none, fill=black!63},
%     bar height=0.6,
%     bar top shift=0.2,
%     group top shift=0.4,
%     group height=0.2,
%     group peaks width=0.2,
%     group peaks height=0.2,
%     group peaks tip position=0,
%     group label node/.append style={align=left}
%   ]{2024-05-01}{2024-07-31}
%
%   \gantttitlecalendar{month=shortname} \\
%
%   \ganttgroup{Phase 1}{2024-05-01}{2024-05-31} \\
%   \ganttbar{Task 1}{2024-05-01}{2024-05-15} \\
%   \ganttbar{Task 2}{2024-05-16}{2024-05-31} \\
%
%   \ganttgroup{Phase 2}{2024-06-01}{2024-06-30} \\
%   \ganttbar{Task 3}{2024-06-01}{2024-06-15} \\
%   \ganttbar{Task 4}{2024-06-16}{2024-06-30} \\
%
% \end{ganttchart}
% \end{sidewaysfigure}

% \newgantttimeslotformat{stardate}{%
%   \def\decomposestardate##1.##2\relax{%
%     \def\stardateyear{##1}\def\stardateday{##2}%
%   }%
%   \decomposestardate#1\relax%
%   \pgfcalendardatetojulian{\stardateyear-01-01}{#2}%
%   \advance#2 by-1\relax%
%   \advance#2 by\stardateday\relax%
% }

% \begin{ganttchart}[
%   hgrid,
%   vgrid,
%   time slot format=stardate
%   ]{2259.55}{2259.67}
%   \gantttitlecalendar{year, month=name, week} \\
% \end{ganttchart}

% \begin{ganttchart}[
%     hgrid style/.style={black, dotted},
%     vgrid={*6{black,dotted}, *1{black, dashed}},
%     x unit=3mm,
%     y unit chart=9mm,
%     y unit title=12mm,
%     time slot format=isodate,
%     group label font=\bfseries \Large,
%     link/.style={->, thick}
%     ]{2024-01-01}{2024-06-30}
%
%     % Headers
%     \gantttitlecalendar{year, month=name, week}\\
%
%     % Groupe Analyse des besoins
%     \ganttgroup[
%         group/.append style={fill=orange}
%     ]{Analyse des besoins}{2024-01-01}{2024-02-28} \\ [grid]
%     \ganttorangebar[
%         name=EtudeFonctionnalites
%     ]{Étude des fonctionnalités}{2024-01-01}{2024-01-31} \\ [grid]
%     \ganttorangebar[
%         name=IdentificationExigences
%     ]{Identification des exigences}{2024-01-15}{2024-02-28} \\ [grid]
%
%     % Groupe Conception
%     \ganttgroup[
%         group/.append style={fill=blue}
%     ]{Conception}{2024-02-01}{2024-03-31} \\ [grid]
%     \ganttbluebar[
%         name=ConceptionArchitecture
%     ]{Conception de l’architecture}{2024-02-01}{2024-02-28} \\ [grid]
%     \ganttbluebar[
%         name=ConceptionInterface
%     ]{Conception de l’interface}{2024-03-01}{2024-03-31} \\ [grid]
%
%     % Groupe Développement
%     \ganttgroup[
%         group/.append style={fill=green}
%     ]{Développement}{2024-03-01}{2024-04-30} \\ [grid]
%     \ganttgreenbar[
%         name=ImplementationFonctionnalites
%     ]{Implémentation des fonctionnalités}{2024-03-01}{2024-03-31} \\ [grid]
%     \ganttgreenbar[
%         name=DeveloppementFonctionnalites
%     ]{Développement des fonctionnalités}{2024-04-01}{2024-04-30} \\ [grid]
%
%     % Groupe Tests
%     \ganttgroup[
%         group/.append style={fill=red}
%     ]{Tests}{2024-04-01}{2024-05-31} \\ [grid]
%     \ganttredbar[
%         name=TestsUnitaires
%     ]{Tests unitaires et d’intégration}{2024-04-01}{2024-04-30} \\ [grid]
%     \ganttredbar[
%         name=TestsPerformance
%     ]{Tests de performance}{2024-05-01}{2024-05-31} \\ [grid]
%
%     % Groupe Mise en production
%     \ganttgroup[
%         group/.append style={fill=purple}
%     ]{Mise en production}{2024-05-01}{2024-05-31} \\ [grid]
%     \ganttpurplebar[
%         name=ConfigurationServeurs
%     ]{Configuration des serveurs}{2024-05-01}{2024-05-15} \\ [grid]
%     \ganttpurplebar[
%         name=PublicationApplications
%     ]{Publication des applications}{2024-05-16}{2024-05-31} \\ [grid]
%
%     % Groupe Formation
%     \ganttgroup[
%         group/.append style={fill=brown}
%     ]{Formation}{2024-05-15}{2024-06-30} \\ [grid]
%     \ganttbrownbar[
%         name=FormationUtilisateurs
%     ]{Formation des utilisateurs}{2024-05-15}{2024-05-31} \\ [grid]
%     \ganttbrownbar[
%         name=DocumentationComplete
%     ]{Documentation complète}{2024-06-01}{2024-06-30} \\ [grid]
%
%     % Groupe Maintenance
%     \ganttgroup[
%         group/.append style={fill=gray}
%     ]{Maintenance}{2024-06-01}{2024-06-30} \\ [grid]
%     \ganttgraybar[
%         name=SurveillanceContinue
%     ]{Surveillance continue}{2024-06-01}{2024-06-15} \\ [grid]
%     \ganttgraybar[
%         name=MisesAJour
%     ]{Mises à jour régulières}{2024-06-16}{2024-06-30} \\ [grid]
%
%     % Links
%     \ganttlink[link mid=0.75]{EtudeFonctionnalites}{ConceptionArchitecture}
%     \ganttlink[link mid=0.75]{IdentificationExigences}{ConceptionInterface}
%     \ganttlink[link mid=0.75]{ConceptionArchitecture}{ImplementationFonctionnalites}
%     \ganttlink[link mid=0.75]{ConceptionInterface}{DeveloppementFonctionnalites}
%     \ganttlink[link mid=0.75]{ImplementationFonctionnalites}{TestsUnitaires}
%     \ganttlink[link mid=0.75]{DeveloppementFonctionnalites}{TestsPerformance}
%     \ganttlink[link mid=0.75]{TestsUnitaires}{ConfigurationServeurs}
%     \ganttlink[link mid=0.75]{TestsPerformance}{PublicationApplications}
%     \ganttlink[link mid=0.75]{PublicationApplications}{FormationUtilisateurs}
%     \ganttlink[link mid=0.75]{FormationUtilisateurs}{DocumentationComplete}
%     \ganttlink[link mid=0.75]{DocumentationComplete}{SurveillanceContinue}
%     \ganttlink[link mid=0.75]{SurveillanceContinue}{MisesAJour}
%
% \end{ganttchart}

\begin{figure}[H]
  \centering
  \includegraphics[width=1\textwidth]{capture/gantt.png}
  \caption{Diagramme de Gantt}
\end{figure}



\section{Estimation des charges}
L’estimation des charges implique de calculer le temps et les ressources
nécessaires pour accomplir chacune des tâches du projet.
Des estimations précises sont essentielles pour la planification budgétaire et
la gestion des ressources humaines.


\begin{table}[htbp]
  \centering
  \begin{tabular}{|l|c|c|}
    \hline
    \textbf{Outil et besoin} & \textbf{Quantité} & \textbf{Prix} \\ \hline
    \LaTeX & 1 & 0 FCFA  \\ \hline
    TypeScript & 1 & 0 FCFA  \\ \hline
    Next.js & 1 & 0 FCFA  \\ \hline
    Capacitor & 1 & 0 FCFA  \\ \hline
    PostgreSQL & 1 & 0 FCFA  \\ \hline
    Docker & 1 & 0 FCFA  \\ \hline
    Git & 1 & 0 FCFA  \\ \hline
    GitHub & 2 & 0 FCFA  \\ \hline
    Neovim & 1 & 0 FCFA  \\ \hline
    VSCodium & 1 & 0 FCFA  \\ \hline
    Figma & 1 & 0 FCFA  \\ \hline
    PC Portable & 2 & 400 000 FCFA  \\ \hline
    Connexion internet & 6 & 25 000 FCFA \\ \hline
    Réalisateurs & 2 & 4 000 000 FCFA \\ \hline
    \multicolumn{2}{|l|}{\textbf{Total}} & \textbf{8 950 000 FCFA} \\ \hline
  \end{tabular}
  \caption{Estimation des charges}
\end{table}

\chapter{Les outils et techniques utilisés}
Ce chapitre présente les outils et les techniques utilisées pour développer
et gérer le projet.

\section{Présentation des techniques}
Les techniques utilisées dans le cadre de ce projet sont les suivantes :

  \textbf{ {Développement Web et Mobile}}

    Pour le développement de la plateforme, nous avons utilisé des techniques
    modernes adaptées au web et aux applications mobiles.
    \begin {itemize}
  \item \textbf{{Next.js}}
    Next.js est un framework React qui permet de créer des applications web côté
    serveur (SSR) et des applications statiques (SSG). Il offre des fonctionnalités
    avancées telles que le rendu côté serveur, le pré-rendu statique, et une
    gestion automatique des routes, ce qui améliore la performance et le
    SEO des applications web.

  \item \textbf{{Capacitor}}
    Capacitor crée un environnement de développement pour les applications en
    transformant les applications web en applications mobiles natives avec
    JavaScript, HTML et CSS.

\end{itemize}

\textbf{{Base de Données}}

  Pour stocker les données de l'application, nous avons utilisé une base de
  données relationnelle PostgreSQL, qui offre des fonctionnalités avancées
  telles que la conformité ACID, la gestion des transactions, et la prise en
  charge de données complexes.

\textbf{{Sécurité}}

  La sécurité est un aspect crucial du développement de la plateforme. Nous
  avons mis en place plusieurs techniques de sécurité pour protéger les données
  des utilisateurs.
  \begin{itemize}
    \item \textbf{{Authentification et Autorisation}}

      Nous avons utilisé un système de session lié à la base de donnée pour
      l'authentification. Les autorisation, pour sécuriser les accès
      à l'application et aux API sont fait par rapport à ces sessions.

    \item \textbf{{Chiffrement des Données}}

      Toutes les communications entre les clients et le serveur sont chiffrées
      en utilisant TLS (Transport Layer Security), garantissant que les données
      transmises restent confidentielles et intactes.
  \end{itemize}

\textbf{Gestion de Projet}

Pour la gestion du projet, nous avons utilisé la méthode objet, qui permet
d'organiser les tâches en fonction de leur importance et de leur priorité.


\section{Présentation des outils}
Les outils utilisés dans le cadre de ce projet sont les suivants :

\textbf{Outils de Développement}
\begin{itemize}
  \item \textbf{Neovim}

    Neovim est un éditeur de texte puissant et extensible, adapté au
    développement de logiciels. Il offre des fonctionnalités avancées telles
    que la coloration syntaxique, la complétion automatique, et la gestion
    des plugins, ce qui améliore la productivité des développeurs.

    \begin{figure}[H]
      \centering
      \includegraphics[width=0.5\textwidth]{images/Neovim-logo.png}
      \caption{Logo Neovim}
    \end{figure}

  \item \textbf{VSCodium}

    VSCodium est un éditeur de code open-source basé sur Visual Studio Code,
    \begin{figure}[H]
      \centering
      \includegraphics[width=1.0in, height=1.0in]{images/codium_cnl.png}
      \caption{Logo VSCodium}
    \end{figure}
\end{itemize}

\textbf{Frameworks et Bibliothèques}
\begin{itemize}
  \item \textbf{React}

    React est la bibliothèque JavaScript principale utilisée pour construire
    les interfaces utilisateur de la plateforme web.

  \item \textbf{Capacitor}

    Capacitor est un framework qui permet de créer des applications mobiles
    multiplateformes avec des technologies web comme React,  Vue et bien d'autre.
    Il offre des fonctionnalités telles que l'accès aux API natives, la gestion
    des plugins, et la compatibilité avec les stores d'applications.

    \begin{figure}[H]
      \centering
      \includegraphics[width=0.5\textwidth]{images/Capacitor-JS-2375506976.png}
      \caption{Logo Capacitor}
    \end{figure}

  \item \textbf{Next.js}

    Next.js est le framework React utilisé pour le développement côté serveur
    et la génération de sites statiques de la partie web de la plateforme.

    \begin{figure}[H]
      \centering
      \includegraphics[width=0.5\textwidth]{images/Nextjs-logo.svg.png}
      \caption{Logo Next.js}
    \end{figure}

  % \item \textbf{Expo}
  %
  %   Expo est la plateforme utilisée pour le développement et le déploiement
  %   des applications mobiles, en fournissant un ensemble d'outils et de
  %   services pour React Native.
  %
  %   \begin{figure}[H]
  %     \centering
  %     \includegraphics[width=0.5\textwidth]{images/logo-wordmark.png}
  %     \caption{Logo Expo}
  %   \end{figure}

  \item \textbf{shadcn/ui}

    shadcn/ui est une bibliothèque de composants React réutilisables, qui
    permet de créer des interfaces utilisateur cohérentes et esthétiques.
\end{itemize}

\textbf{Outils de Contrôle de Version}
\begin{itemize}
  \item \textbf{Git}

    Git est le système de contrôle de version distribué utilisé pour suivre
    les modifications du code source et faciliter la collaboration entre les
    développeurs.

    \begin{figure}[H]
      \centering
      \includegraphics[width=0.3\textwidth]{images/Git-logo.svg.png}
      \caption{Logo Git}
    \end{figure}

  \item \textbf{GitHub}

    GitHub est la plateforme utilisée pour héberger le code source et faciliter
    la collaboration via des fonctionnalités comme les pull requests et les
    revues de code.

    \begin{figure}[H]
      \centering
      \includegraphics[width=1.0in, height=1.0in]{images/GitHub_Invertocat_Logo.svg.png}
      \caption{Logo GitHub}
    \end{figure}
\end{itemize}

\textbf{Outils de Conception}
\begin{itemize}
  \item \textbf{Figma}

    Figma est l'outil de conception utilisé pour créer les maquettes et les
    prototypes de l'interface utilisateur de la plateforme, en facilitant la
    collaboration et le partage des designs.

    \begin{figure}[H]
      \centering
      \includegraphics[width=1.0in, height=1.2in]{images/800px-Figma-logo.svg.png}
      \caption{Logo Figma}
    \end{figure}

  \item \textbf{\LaTeX}

    \LaTeX est un système de composition de documents utilisé pour rédiger les
    documentations techniques et les rapports. Il offre une mise en
    page professionnelle et une gestion avancée des références. Il permet
    également de créer des diagrammes UML et de toute sorte.

    \begin{figure}[H]
      \centering
      \includegraphics[width=0.5\textwidth]{images/LaTeX_project_logo_bird.svg.png}
      \caption{Logo \LaTeX}
    \end{figure}
\end{itemize}

\textbf{Outils de Base de Données}
\begin{itemize}
  \item \textbf{PostgreSQL}

    PostgreSQL est le système de gestion de base de données relationnelle
    utilisé pour stocker les données de l'application, en offrant des
    fonctionnalités avancées de gestion des données et de sécurité.

    \begin{figure}[H]
      \centering
      \includegraphics[width=1.0in, height=1.0in]{images/Postgresql_elephant.svg.png}
      \caption{Logo PostgreSQL}
    \end{figure}

  \item \textbf{Prisma}

    Prisma est un \ac{ORM}  utilisé pour simplifier
    l'accès à la base de données PostgreSQL et faciliter les opérations de
    lecture et d'écriture des données.
    \begin{figure}[H]
      \centering
      \includegraphics[width=1.0in, height=1.0in]{images/icons8-prisma-orm-500.png}
      \caption{Logo Prisma}
    \end{figure}

  \item \textbf{tRPC}

    tRPC est une bibliothèque de communication entre le client et le serveur
    qui facilite l'appel des API et la gestion des données de manière
    sécurisée et efficace.

    \begin{figure}[H]
      \centering
      \includegraphics[width=0.5\textwidth]{images/trpc.png}
      \caption{Logo tRPC}
    \end{figure}
\end{itemize}

\textbf{Outils de Sécurité}
\begin{itemize}

  \item \textbf{TLS (Transport Layer Security)}

    TLS est utilisé pour chiffrer les communications entre les clients et le
    serveur, en garantissant la confidentialité et l'intégrité des
\end{itemize}

% \item \textbf{Outils de test}
%   \begin{itemize}
%     \item \textbf{Jest}
%
%       Jest est utilisé pour les tests unitaires et les tests d'intégration du
%       code JavaScript et TypeScript.
%
%     \item \textbf{React Testing Library}
%
%       React Testing Library est utilisée pour tester les composants React de
%       manière efficace et fiable.
%   \end{itemize}

\textbf{Outils de conteneurisation}
\begin{itemize}
  \item \textbf{Docker}

    Docker est utilisé pour créer des conteneurs légers et portables qui
    encapsulent les applications et leurs dépendances, facilitant le
    déploiement et la gestion des applications dans différents environnements.

    \begin{figure}[H]
      \centering
      \includegraphics[width=0.5\textwidth]{images/Docker_logo.svg.png}
      \caption{Logo Docker}
    \end{figure}
\end{itemize}

\textbf{Outils de Déploiement}
\begin{itemize}
  \item \textbf{Vercel}

    Vercel est utilisé pour le déploiement et l'hébergement de la partie web
    de la plateforme développée avec Next.js.

    \begin{figure}[H]
      \centering
      \includegraphics[width=0.5\textwidth]{images/Vercel_logo_black.svg.png}
      \caption{Logo Vercel}
    \end{figure}

  \item \textbf{AppFlow}

    AppFlow est utilisé pour le déploiement et la gestion des applications
    mobiles développées avec Capacitor, React Native et bien d'autre, en
    facilitant le processus de publication
    sur les stores d'applications.

    \begin{figure}[H]
      \centering
      \includegraphics[width=0.3\textwidth]{images/AppFlow.png}
      \caption{Logo AppFlow}
    \end{figure}
\end{itemize}


\section{Les captures d'écrans}
Les captures d’écran illustrent visuellement les étapes importantes du
développement et l’interface utilisateur du produit final. Elles servent à
documenter le travail accompli et à fournir des exemples concrets de
l’application en action.

\begin{figure}[H]
  \centering
  \includegraphics[width=1\textwidth]{./capture/home.png}
  \caption{Écran d'accueil sur l'application web}
\end{figure}

\begin{figure}[H]
  \centering
  \includegraphics[width=1\textwidth]{capture/login.png}
  \caption{Écran de connexion sur l'application web}
\end{figure}

\begin{figure}[H]
  \centering
  \includegraphics[width=1\textwidth]{capture/signup.png}
  \caption{Écran de création de compte sur l'application web}
\end{figure}

\begin{figure}[H]
  \centering
  \includegraphics[width=1\textwidth]{capture/tree.png}
  \caption{Écran de création d'un arbre généalogique sur l'application web}
\end{figure}

\begin{figure}[H]
  \centering
  \includegraphics[width=1\textwidth]{capture/member.png}
  \caption{Écran d'ajout d'un membre à un généalogique sur l'application web}
\end{figure}

\begin{figure}[H]
  \centering
  \includegraphics[width=1\textwidth]{capture/view.png}
  \caption{Écran de visualisation d'un arbre généalogique}
\end{figure}


\begin{figure}[H]
  \centering
  \includegraphics[width=0.3\textwidth]{./capture/homem.png}
  \caption{Écran d'accueil sur l'application mobile}
\end{figure}


\begin{figure}[H]
  \centering
  \includegraphics[width=0.3\textwidth]{capture/loginw.png}
\caption{Écran de connexion sur l'application mobile}
\end{figure}


\begin{figure}[H]
  \centering
  \includegraphics[width=0.3\textwidth]{capture/signupw.png}
  \caption{Écran de création de compte sur l'application mobile}
\end{figure}


\begin{figure}[H]
  \centering
  \includegraphics[width=0.3\textwidth]{capture/treem.png}
  \caption{Écran de création d'un arbre généalogique sur l'application mobile}
\end{figure}


\begin{figure}[H]
  \centering
  \includegraphics[width=0.3\textwidth]{capture/memberm.png}
  \caption{Écran d'ajout d'un membre à un généalogique sur l'application mobile}
\end{figure}


\chapter*{Conclusion}
\addcontentsline{toc}{chapter}{Conclusion}
\label{chap:conclusion}
En somme, notre projet de création d’une plateforme web et mobile dédiée à la
généalogie collaborative répond à un besoin croissant de préservation et de
partage de l’héritage familial à l’ère numérique. Il s’inscrit dans une
dynamique de démocratisation de l’accès à la généalogie, permettant à chacun
d’explorer ses racines et de se connecter à son histoire familiale,
indépendamment de ses connaissances préalables ou de sa localisation
géographique.

La méthode objet nous a permis une approche itérative et flexible,
garantissant que la plateforme s’adapte constamment aux besoins des
utilisateurs. Les résultats obtenus jusqu’à présent sont prometteurs et
démontrent l’efficacité de notre approche méthodologique ainsi que la
pertinence des technologies choisies pour le développement de cette solution.

Toutefois, notre travail ne constitue qu’une première étape d’un processus plus
large. Outre les fonctionnalités mises en place, plusieurs questions restent
sans réponse, ouvrant ainsi la porte à de futurs développements. Par exemple :
comment pouvons-nous intégrer des analyses généalogiques avancées pour offrir
des connaissances plus profondes aux utilisateurs? Quelles nouvelles fonctionnalités
pourraient enrichir davantage l’expérience utilisateur, tout en garantissant
la sécurité et la confidentialité des données? Les possibilités sont nombreuses.

En fin de compte, nous espérons que notre plateforme deviendra une référence
incontournable dans le domaine de la généalogie numérique. Elle contribuera
ainsi à la préservation et à la valorisation de l’histoire familiale pour les
générations futures. Ce projet, bien qu’ambitieux, n’est qu’une étape dans la
mission plus large de connecter les individus à leurs racines et de renforcer
le lien familial à travers les outils numériques.


\pagenumbering{alph}

\label{sec:tableofcontents}
\tableofcontents

\printbibliography
\label{sec:bibliographie}

\chapter*{Webographie}
\label{sec:webographie}
\addcontentsline{toc}{chapter}{Webographie}
\begin{itemize}
    \item \textbf{Architecture distribuée} \\
    Lien: \url{http://fr.wikipedia.org/w/index.php?title=Architecture_distribu%C3%A9e&oldid=212787328} \\
    Description: Article Wikipédia (encyclopédie libre). \\
    Consulté le : 12-Avril-2024 à 15h00

    \item \textbf{TypeScript Documentation} \\
    Lien: \url{https://www.typescriptlang.org/docs/} \\
    Description: Documentation officielle pour TypeScript, un sur-ensemble typé de JavaScript qui se compile en JavaScript pur. \\
    Consulté le : 16-février-2024 à 09h30

    \item \textbf{Next.js Documentation} \\
    Lien: \url{https://nextjs.org/docs} \\
    Description: Documentation officielle pour Next.js. \\
    Consulté le : 16-février-2024 à 15h45

    \item \textbf{Capacitor} \\
    Lien: \url{https://capacitorjs.com/docs} \\
    Description: Documentation officielle pour Capacitor. \\
    Consulté le : 16-février-2024 à 17h00

    \item \textbf{Prisma Getting Started} \\
    Lien: \url{https://www.prisma.io/docs/getting-started} \\
    Description: Guide de démarrage pour Prisma. \\
    Consulté le : 18-février-2024 à 16h15

    \item \textbf{LaTeX Project Documentation} \\
    Lien: \url{https://www.latex-project.org/help/documentation/} \\
    Description: Documentation officielle du projet LaTeX. \\
    Consulté le : 16-Janvier-2024 à 18h30

    \item \textbf{Lucia Auth Documentation} \\
    Lien: \url{https://lucia-auth.com/} \\
    Description: Documentation pour Lucia. \\
    Consulté le : 03-Mai-2024 à 14h45

    \item \textbf{shadcn/ui Documentation} \\
    Lien: \url{https://ui.shadcn.com/docs} \\
    Description: Documentation pour shadcn/ui. \\
    Consulté le : 16-février-2024 à 19h00

    \item \textbf{Docker Guides} \\
    Lien: \url{https://docs.docker.com/guides/} \\
    Description: Guides et documentation officielle pour Docker. \\
    Consulté le : 27-février-2024 à 19h22

    % \item \textbf{Git Documentation} \\
    % Lien: \url{https://git-scm.com/doc} \\
    % Description: Documentation officielle de Git. \\
    % Consulté le : 16-février-2024 à 12h21

% \item \textbf{TikZ-UML User Guide} \\
    % Lien: \url{https://perso.ensta-paris.fr/~kielbasi/tikzuml/var/files/html/web-tikz-uml-userguide.html} \\
    % Description: Guide utilisateur pour TikZ-UML, une extension de TikZ pour dessiner des diagrammes UML en LaTeX. \\
    % Consulté le : 25-février-2024 à 17h45
\end{itemize}

\end{document}

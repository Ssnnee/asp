\part{Analyse et Conception }
\chapter{Etude préliminaire et Méthode}

Dans ce chapitre, nous allons présenter l’étude préliminaire de notre projet de
création collaborative et de partage d’arbres généalogiques, ainsi que la
méthodologie que nous allons suivre pour sa conception et son développement.
Nous commencerons par une analyse approfondie de l’existant. Nous décrirons la
situation actuelle de \firm dans le domaine de la généalogie en ligne, nous
identifierons ses limites et nous proposerons des solutions pour les surmonter.

Ensuite, nous présenterons la méthode que nous allons utiliser pour concevoir notre
plateforme. Nous mettrons l’accent sur le langage de modélisation \ac{UML} ,
ainsi que sur les différents types de diagrammes UML que nous utiliserons pour
représenter notre système de manière visuelle et structurée. Nous discuterons
également des processus de développement que nous allons adopter, notamment
le \ac{UP} et le \ac{2TUP}. Nous verrons comment ils peuvent être intégrés
à UML pour assurer une gestion efficace du projet.

Ce chapitre posera les bases nécessaires à la réalisation de notre projet en
fournissant une compréhension approfondie de son contexte et en établissant
les lignes directrices pour sa conception et son développement. En combinant
une analyse approfondie de l’existant avec une méthodologie de conception
rigoureuse, nous sommes bien placés pour créer une plateforme généalogique
collaborative et innovante qui répondra aux besoins de nos utilisateurs tout
en garantissant sa robustesse et sa pérennité.

\section{Etude de l'existant}
\subsection{Description des activités (la situation actuelle)}
Actuellement, \firm ne dispose pas d’une plateforme dédiée à la création
collaborative et au partage d’arbres généalogiques. Les activités liées à la
généalogie sont principalement effectuées de manière traditionnelle, impliquant
la collecte de documents papier, la communication verbale et l’échange de fichiers
numériques par courrier électronique ou d’autres moyens non structurés. Les
membres de la famille rencontrent souvent des difficultés pour collaborer
efficacement à la création et au partage d'arbres généalogiques en raison de
l'absence d'un outil centralisé et convivial.

\subsection{Critique de l'existant (les limites)}
Cette approche traditionnelle présente plusieurs limites. Elle rend difficile
la collaboration entre les membres de la famille, en particulier ceux qui sont
éloignés géographiquement, et ne permet pas un partage facile et sécurisé des
informations généalogiques. De plus, la préservation à long terme de l’histoire
familiale est compromise, car les documents papier peuvent se perdre ou se
détériorer avec le temps, et les fichiers numériques peuvent être dispersés et mal organisés.

\subsection{Proposition de solutions}
Pour surmonter ces défis, nous proposons les solutions suivantes :
\begin{itemize}
  \item  Utiliser des logiciels de généalogie hors ligne : cette solution implique
    l’installation de logiciels sur des ordinateurs personnels pour créer et gérer
    des arbres généalogiques. Cependant, cela peut présenter certaines limites
    en termes de collaboration et de partage, car les données sont souvent stockées
    localement et ne permettent pas une collaboration facile entre les membres de la famille.

  \item  Utiliser des sites Web de généalogie : cela consiste à
    utiliser des sites Web spécialisés en généalogie pour créer et partager des arbres
    généalogiques. Cette méthode est pratique, car elle permet de partager ses arbres
    avec d’autres utilisateurs. Cependant, les fonctionnalités des sites sont limitées et ne
    permettent pas toujours une collaboration efficace.

  \item Développer d’une plateforme web et mobile dédiée à la création
    collaborative et au partage d’arbres généalogiques : Cette solution propose
    de créer une plateforme personnalisée qui répond spécifiquement aux besoins de
    \firm Elle permettra aux utilisateurs de collaborer à la création d’arbres
    généalogiques, de partager facilement des informations avec leur famille et
    leurs proches, et offrira des fonctionnalités avancées, notamment la gestion
    de la confidentialité des données et des outils d’analyse.

\end{itemize}

\subsection{Choix de la solution}
Parmi les solutions présentées, nous choisissons de développer une plateforme web
et mobile dédiée à la création collaborative et au partage d’arbres généalogiques.
Elle présente la meilleure combinaison de convivialité, de fonctionnalités avancées
et de contrôle sur la confidentialité des données. En offrant une plateforme
centralisée et sécurisée, nous pouvons garantir une expérience utilisateur optimale
tout en répondant aux besoins diversifiés des utilisateurs.

En résumé, notre choix reflète notre engagement à fournir une solution
efficace et innovante pour relever les défis de \firm.

